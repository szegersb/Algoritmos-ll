\documentclass[10pt,a4paper]{article}

\input{AEDmacros}
\usepackage{caratula} % Version modificada para usar las macros de algo1 de ~> https://github.com/bcardiff/dc-tex


\titulo{Trabajo Pr\'actico 1: Especificaci\'on y WP}
\subtitulo{Fondo Monetario Com\'un}

\fecha{\today}

\materia{Algoritmos y Estructuras de Datos}
\grupo{Grupo Debuggers}

\integrante{Zegers, Santiago}{1433/21}{zegerssantiagob@gmail.com}
\integrante{Azcurra, Mariano}{1321/21}{mariano.azcurra@hotmail.es}
\integrante{Gonzalez Villagra, Nicolás Andrés}{1545/21}{nicofcen86@gmail.com}
\integrante{Basualdo, Camilo}{225/19}{camilobasualdo@gmail.com}
% Pongan cuantos integrantes quieran

% Declaramos donde van a estar las figuras
% No es obligatorio, pero suele ser comodo
\graphicspath{{../static/}}

\begin{document}

\maketitle

\begin{align*}
\maketitle

% EJERCICIO 1
\section{Especificación}
\subsection{redistribuciónDeLosFrutos:}
\begin{proc}{redistribuciónDeLosFrutos}{\In recursos: \TLista{\float}, \In cooperan: \TLista{\bool}}{\TLista{\float}}
\vspace{0.1cm}	
    \requiere{\longitud{cooperan}>0 \y {\longitud{recursos}} = {\longitud{cooperan}} \y todosPositivos(recursos)}
\vspace{0.1cm}
	\asegura{{\longitud{result}} = {\longitud{recursos}} \y recursosLuegoDeDistribución(result, recursos, cooperan)} 
\end{proc}
\vspace{0.3cm}

\pred{todosPositivos}{s: \seqr}{\paraTodo[unalinea]{i}{\ent}{0\leq i < \longitud{s}\implicaLuego s[i] \geq 0}}
\vspace{0.3cm}

\pred{recursosLuegoDeDistribución}{r_f: \seqr, r_i: \seqr, b: \seqB}{
\paraTodo[unalinea]{i}{\ent}{0\leq i < \longitud{r_i} \yLuego aporta(i, b) \implicaLuego r_f[i] = redistribución(r_i,b)} \y 
\newline{\paraTodo[unalinea]{j}{\ent}{0\leq j < \longitud{r_i} \yLuego \neg aporta(j, b) \implicaLuego r_f[j] = r_i[j] + redistribución(r_i,b)}} }
\vspace{0.3cm}

\pred{aporta}{i: \ent, b: \seqB}{
b[i] = true
}
\vspace{0.3cm}

\aux{redistribución}{s: \seqr, b: \seqB}{\float}{(\displaystyle\sum_{i=0}^{\longitud{b}-1} \IfThenElse{b[i]=true}{s[i]}{0})/\longitud{b}}
\vspace{0.3cm}

%EJERCICIO 2
\subsection{trayectoriaDeLosFrutosIndividualesALargoPlazo}
\begin{proc}{trayectoriaDeLosFrutosIndividualesALargoPlazo}{\Inout trayectorias: \TLista{\seqr}, \In cooperan: \TLista{\bool}, \In apuestas: \TLista{\seqr}, \In pagos: \TLista{\seqr}, \In eventos: \TLista{\TLista{\nat}}}{}
\vspace{0.1cm}
    \requiere{trayectorias = trayectorias_{0} \y mismaLongitud(trayectorias, cooperan, apuestas, pagos, eventos) 
    \vspace{0.1cm}
    \newline{\y \longitud{cooperan}>0 \y mismasApuestasQuePagos(apuestas, pagos) \y sonPositivos(pagos)} 
    \vspace{0.1cm}
    \newline{\y eventosValidos(eventos, apuestas)} 
    \vspace{0.1cm}
    \y comienzaSoloConRecursoInicial(trayectorias) \wedge
    sonApuestasValidas(apuestas)
    }
\vspace{0.1cm}
	\asegura{{\longitud{trayectorias}} = {\longitud{trayectorias_{0}}} \y contieneALosIniciales(trayectorias, trayectorias0) \y 
    \vspace{0.1cm}
    \newline{tieneLosEventos(trayectorias, eventos) \y trayectoriasIndividuales(trayectorias, cooperan, apuestas, pagos, eventos)}}
\end{proc}
\vspace{0.3cm}

\pred{mismaLongitud}{t, c, a, p, e: \TLista{T}}{
\longitud{t} = \longitud{c} \y \longitud{c} = \longitud{a} \y
\newline{\longitud{a} = \longitud{p} \y \longitud{p} = \longitud{e}}
}
\vspace{0.3cm}

\pred{mismasApuestasQuePagos}{a,p: \TLista{\seqr}}{
\paraTodo[unalinea]{i}{\ent}{0\leq i < \longitud{a} \implicaLuego \longitud{a[i]} = \longitud{p[i]}}
}
\vspace{0.3cm}

\pred{sonPositivos}{s1: \TLista{\seqr}}{
\paraTodo[unalinea]{i}{\ent}{0\leq i < \longitud{s1} \implicaLuego todosPositivos(s1[i])}
}
\vspace{0.3cm}

\pred{eventosValidos}{e: \TLista{\TLista{\nat}}, p: \TLista{\seqr}}{
\paraTodo[unalinea]{i}{\ent}{0\leq i < \longitud{e} } \implicaLuego{\paraTodo[unalinea]{j}{\ent}{0\leq j < \longitud{e[i]} \implicaLuego 0 \leq e[i][j] < \longitud{p[i]}}}
}
\vspace{0.3cm}

\pred{comienzaSoloConRecursoInicial}{t: \TLista{\seqr}}{
\paraTodo[unalinea]{i}{\ent}{0\leq i < \longitud{t} \yLuego sonPositivos(t) \implicaLuego \longitud{t[i]} = 1}
}
\vspace{0.3cm}

\pred{contieneALosIniciales}{t: \TLista{\seqr}, t0: \TLista{\seqr}}{
\paraTodo[unalinea]{i}{\ent}{0\leq i < \longitud{t} \implicaLuego t[i][0] = t0[i][0]}
}
\vspace{0.3cm}

\pred{tieneLosEventos}{t: \TLista{\seqr}, e:\TLista{\TLista{\nat}}}{
\paraTodo[unalinea]{i}{\ent}{0\leq i < \longitud{t} \implicaLuego \longitud{t[i]} - 1 = \longitud{e[i]}}
}
\vspace{0.3cm}

\pred{trayectoriasIndividuales}{t: \TLista{\seqr}, c: \TLista{\bool}, a: \TLista{\seqr}, p: \TLista{\seqr}, e: \TLista{\TLista{\nat}}}{
\paraTodo[unalinea]{i}{\ent}{0\leq i < \longitud{t} \implica trayectoria(i, t[i], t, c, a, p, e)}
}
\vspace{0.3cm}

\pred{trayectoria}{i:\ent, r:\seqr , t: \TLista{\seqr}, c: \TLista{\bool}, a: \TLista{\seqr}, p: \TLista{\seqr}, e: \TLista{\TLista{\nat}}}{
\paraTodo[unalinea]{j}{\ent}{1\leq j < \longitud{r} \implica (\neg aporta(i, c) \implicaLuego r[j] = r[j-1] * pagoEvento(i, j, a, p, e) + redistribucionEnJ(j, t, c)) \vee (aporta(i, c) \implicaLuego r[j] = redistribucionEnJ(j, t, c))} 
}
\vspace{0.3cm}

\pred{sonApuestasValidas}{apuestas: \TLista{\TLista{\float}}}{(\forall j: \ent) (\forall k: \ent) (0 \leq j < |\text{apuestas}| \land 0 \leq k < |\text{apuestas}[j]| \implicaLuego 0 \leq \text{apuestas}[j][k] \leq 1)}

\aux{pagoEvento}{i,j: \ent, a,p: \TLista{\seqr}, e: \TLista{\TLista{\nat}}}{\float}{
apuestas[i][evento[i][j-1]] * pagos[i][evento[i][j-1]]
}
\vspace{0.3cm}

\aux{redistribucionEnJ}{j: \ent, t: \TLista{\seqr}, c: \TLista{\bool}}{\float}{
(\displaystyle\sum_{i=0}^{\longitud{b}-1} \IfThenElse{c[i]=true}{t[i][j]}{0})/\longitud{c}
}
\vspace{0.3cm}

% EJERCICIO 3
\subsection{trayectoriaExtrañaEscalera}
\begin{proc}{trayectoriaExtrañaEscalera}{\In trayectoria: \seqr}{\bool}
\vspace{0.1cm}	
   % \requiere{{\longitud{trayectoria}} > 2}
   \requiere{{\longitud{trayectoria}} > 1}
\vspace{0.1cm}
	\asegura{result = true \iff (|trayectoria| > 2 \y hayUnMaximoLocal(trayectoria)) \lor (|trayectoria| = 2 \y trayectoria[0] \neq trayectoria[1])} 
\end{proc}
\vspace{0.3cm}

\pred{hayUnMaximoLocal}{t: \seqr}{
\existe[unalinea]{i}{\ent}{1\leq i < \longitud{t} \yLuego (s[i] > s[i-1] \y s[i] > s[i+1])} \y
\neg{\existe[unalinea]{j}{\ent}{1\leq j < \longitud{t} \y i\neq j\yLuego (s[j] > s[j-1] \y s[j] > s[j+1])}}
}

% EJERCICIO 4
\subsection{individuoDecideSiCooperarONo}
\begin{proc}{individuoDecideSiCooperarONo}{\In individuo: \nat, \In recursos: \seqr, \Inout cooperan: \seqB, \In apuestas: \TLista{\seqr}, \In pagos: \TLista{\seqr}, \In eventos: \TLista{\TLista{\nat}}}{}
\vspace{0.1cm}	
    \requiere{
    0 \leq individuo < |recursos| \wedge cooperan = cooperan_{0} \wedge 
    \newline{|cooperan| > 0 \y mismaLongitud(recursos, cooperan, apuestas, pagos, eventos) \wedge }
     
\newline{\y mismasApuestasQuePagos(apuestas, pagos) \y sonPositivos(pagos)} 
    \vspace{0.1cm}
    \newline{\y eventosValidos(eventos, apuestas)} \wedge 
    sonApuestasValidas(apuestas)
    }
\vspace{0.1cm}
	\asegura{\longitud{cooperan} = \longitud{cooperan_{0}} \wedge \\ \paraTodo[unalinea]{i}{\ent}{0 \leq i < |cooperan_0| \wedge i \neq individuo \implicaLuego
        cooperan[i] = cooperan_0[i]}{} \wedge \\ 
        ganancias(individuo, apuestas[individuo], eventos[individuo], pagos[individuo], cooperan, recursos) \geq 
\newline{ganancias(individuo, apuestas[individuo], eventos[individuo], pagos[individuo], cooperan_{0}, 
recursos)})}
        }
\end{proc}
\vspace{0.3cm}


% EJERCICIO 5
\subsection{individuoActualizaApuesta}
\begin{proc}{individuoActualizaApuesta}{
\In individuo: \nat,
\In recursos: \TLista{\float}, 
\In cooperan: \TLista{\bool}, 
\Inout apuestas: \TLista{\TLista{\float}}, 
\In pagos:  \TLista{\TLista{\float}},
\In eventos: \TLista{\TLista{\nat}}}
{}
	%    \modifica{parametro1, parametro2,..}
	\requiere{apuestas = A_{0} \y sonApuestasValidas(A_{0})\hspace{0.03cm} \y |cooperan| > 0
 \newline{sonPositivos(pagos) \y eventosValidos(eventos, apuestas)} \y \\ 0 \leq individuo < |recursos| \wedge mismaLongitud(recursos, cooperan, apuestas, eventos) }
	\asegura{|apuestas| = |A_{0}| \wedge \\ sonApuestasValidas(apuestas) \wedge \\ 
 \begin{minipage}[t]{\textwidth}
 \paraTodo[unalinea]{j}{\ent}{0 \leq j < |apuestas[individuo]| \wedge j \neq individuo \implicaLuego  apuestas [j] =  A_{0}[j]} 
 \end{minipage} 
 \\ \wedge 
 \neg\existe{apuestaTeorica}{\TLista{\float}}{esApuestaValida(apuestaTeorica) \\
  \yLuego
 ganancias (individuo, apuestaTeorica, eventos[individuo], pagos[individuo],cooperan, recursos) > \\  ganancias (individuo, apuestas[individuo], eventos[individuo], pagos[individuo],cooperan, recursos
 )}
}
\end{proc}
\vspace{5mm}

	\aux{ganancias}{individuo: \nat, 
 apuesta: \TLista{\float}, 
 eventos: \TLista{\nat}, pagos: \TLista{\float}, cooperan: \TLista{Bool}, recursos: {\seqr}}{\float}{\\\IfThenElse{cooperan[individuo]}{\\redistribución (recursos, cooperan)}{\\performanceIndividual(apuesta, pagos, individuo, eventos, recursos[individuo]) +\\ redistribución (recursos, cooperan)
}}
 
 \vspace{5mm}


 \aux{performanceIndividual}{apuestas:\TLista{\float}, pagos: \TLista{\float}, Individuo: \nat, eventos[individuo]: \seqr, recursos: \float}{\float}{\\recursos * \prod_{j=0}^{|apuestas|-1} pagos[j]*apuestas[j]^{#apariciones(eventos, j+1)}}
 \vspace{5mm}


 \vspace{5mm}
\pred{esApuestaValida}{apuesta: \TLista{\float}}{\paraTodo[unalinea]{j}{\ent}{0 \leq j < |apuesta| \implicaLuego  (- apuestas [j] \leq 0 \wedge \sum_{j=0}^{|apuesta|-1} apuestas[j] = 1)}}
\vspace{5mm}

% SECCION 2 CORRECTITUD
\section{Demostración de correctitud}

Tenemos la tripla de Hoare como dato, ya que nos dan el \textit{requiere}, el \textit{asegura} y el código que implementa en smallang. Si llamamos al \textit{requiere} P, al \textit{asegura} Q y al código S, tenemos que demostrar que 

 \begin{center}
     \{P\} S \{Q\}
 \end{center}

\\
es correcta respecto de S. Para eso, aplicaremos la siguiente f\'ormula:
 \begin{center}
  $$  \{P\} S \{Q\} \leftrightarrow   \{P\} \implies WP(S,Q) $$
 \end{center}

Empezamos por calcular entonces la WP(S,Q), la cual podemos reescribir de la siguiente forma.

 \begin{center}
   WP(S,Q)= WP(res := recursos; i:= 0, \textbf{while}, Q)
 \end{center}
Por el Axioma 1, tenemos lo siguiente

 \begin{center}
   WP(S,Q)= WP(res := recursos; i:= 0, WP(\textbf{while}, Q))
 \end{center}

Para demostrar que el ciclo es parcialmente correcto, aplicamos el Teorema del Invariante ya que es un ciclo, debemos probar lo siguiente

\begin{itemize}
	\item $P_{c} \implies I$
	\item $\{I \wedge B\} S1 \{I\}$
	\item $\{I \wedge \neg B\} \implies Q_{c} $
\end{itemize}\\
\vspace{5cm}
Donde  $P_{c}\equiv \{i=0 \wedge res=recursos \wedge apuesta_{c} + apuesta_{s} = 1 ∧ pago_{c} > 0 \wedge pago_{s} > 0 \wedge apuesta_{c} > 0 \wedge apuesta_{s} > 0 \wedge recurso > 0\}$, $B \equiv \{i < |eventos|\}$, $Q_{c}\equiv \{ res = recursos * (apuesta.c * pago.c)^{cantApariciones(eventos,T)}*(apuesta.s * pago.s)^{cantApariciones(eventos,F)}\}$, S1 el c\'odigo del ciclo
 \vspace{5mm}
\begin{lstlisting}[caption={Lo llamaremos S1 en los siguientes pasos.},label=code:for]
	if eventos[i] then
        res = (res x apuesta.c) x pago.c //lo llamaremos S2
    else
        res = (res x apuesta.s) x pago.s //lo llamaremos S3
    endif
    i = i+1
	\end{lstlisting}

e I es el invariante que proponemos:
 \begin{center}
 $I \equiv \{0 \leq i \leq |eventos| \yLuego res = recursos *
 \prod_{j=0}^{i-1} \IfThenElse{eventos[j]}{apuesta.c * pago.c}{apuesta.s * pago.s} \}$
 \end{center}
  \vspace{5mm}
Ahora, podemos comenzar a demostrar los tres puntos del Teorema del Invariante. 
\vspace{5mm}
\\
\underline{$P_{c} \implies I$:}
\vspace{5mm}
\\
Asumiendo $P_{c}$ tenemos luego en I lo siguiente:
\vspace{5mm}
\\
$I \equiv \{0 \leq \textbf{0} \leq |eventos| \yLuego res = recursos *
 \prod_{j=0}^{\textbf{-1}}if... = recursos \} \equiv \{ res = recursos \}$ 
\vspace{5mm}
\\
Por lo que, se demuestra que la precondición del ciclo implica el Invariante, o lo que es lo mismo, el Invariante es válido antes de ingresar al ciclo.
\vspace{5mm}
\\
\underline{$\{I \wedge B\} S1 \{I\}$:}
\vspace{5mm}
\\
Validar esta tripla, implica demostrar que 
 \begin{center}
$\{I \wedge B\} \implies WP(S1,I)$:
 \end{center}

Por el Axioma 3, tenemos que 
 \begin{center}
$WP(S1,I) \equiv WP(\IfThenElse{B}{S2}{S3};i:=i+1,I) \equiv WP(\IfThenElse{B}{S2}{S3},WP(i:=i+1,I))$
 \end{center}
\vspace{5mm}
\\
\textbf{Calculamos $WP(i:=i+1,I)$:}
\vspace{5mm}
\\
Por Axioma 1, tenemos que 
 \begin{center}
$WP(i:=i+1,I)\equiv I_{i+1}^{i}$
 \end{center}
  \begin{center}
$I_{i+1}^{i}\equiv\{0 \leq \textbf{i+1} \leq |eventos| \yLuego res = recursos *
 \prod_{j=0}^{\textbf{i+1-1}} \IfThenElse{eventos[j]}{apuesta.c * pago.c}{apuesta.s * pago.s} \} $
 \end{center}
 Simplificando en la productoria tenemos el resultado de $WP(i:=i+1,I)$:
  \begin{center}
$I_{i+1}^{i}\equiv\{0 \leq \textbf{i+1} \leq |eventos| \yLuego res = recursos *
 \prod_{j=0}^{\textbf{i}} \IfThenElse{eventos[j]}{apuesta.c * pago.c}{apuesta.s * pago.s} \} $
 \end{center}

 Luego, por Axioma 4, 
\vspace{5mm}
\\
$ WP(\IfThenElse{B}{S2}{S3},I_{i+1}^{i})\equiv (B \wedge WP(S2,I_{i+1}^{i})) \lor (\neg B \wedge WP(S3,I_{i+1}^{i}))  $
\vspace{5mm}
\\
\textbf{Calculamos $(B \wedge WP(S2,I_{i+1}^{i}))\vspace{2mm} \\ 
\equiv\{eventos[i] \wedge (I_{i+1}^{i})_{res*apuesta.c*pago.c}^{res}\} \vspace{2mm} \\  
\equiv\{eventos[i] \wedge 0 \leq i+1 \leq |eventos| \yLuego\vspace{2mm} \\  \textbf{res * apuesta.c * pago.c}= recursos  * 
 \prod_{j=0}^{\textbf{i}} \IfThenElse{eventos[j]}{apuesta.c * pago.c}{apuesta.s * pago.s} \}
$}
\vspace{5mm}
\\
Como $eventos[i]=true$ podemos simplificar de productoria este valor, quedando lo siguiente: 
\vspace{3cm}
\\
$(B \wedge WP(S2,I_{i+1}^{i}) 
\vspace{2mm} \\ \equiv\{
eventos[i] \wedge 
(0 \leq i+1 \leq |eventos|) \yLuego \\\textbf{res}= recursos  * 
 \prod_{j=0}^{\textbf{i-1}} \IfThenElse{eventos[j]}{apuesta.c * pago.c}{apuesta.s * pago.s} \}$
\vspace{5mm}
\\

 Análogamente, para el calculo de $\neg B \wedge WP(S3,I_{i+1}^{i})$, obtenemos: 
\vspace{5mm}
\\
$(\neg B \wedge WP(S3,I_{i+1}^{i}))
\vspace{2mm} \\ \equiv \{
\neg eventos[i] \wedge 
(0 \leq i+1 \leq |eventos|) \yLuego \\\textbf{res}= recursos  * 
 \prod_{j=0}^{\textbf{i-1}} \IfThenElse{eventos[j]}{apuesta.c * pago.c}{apuesta.s * pago.s}  \} $
\vspace{5mm}
\\
Por lo que, 
\vspace{5mm}
\\
$ WP(\IfThenElse{B}{S2}{S3},I_{i+1}^{i}) \vspace{2mm} \\ \equiv def(B) \yLuego (B \wedge WP(S2,I_{i+1}^{i})) \lor (\neg B \wedge WP(S2,I_{i+1}^{i})) \vspace{2mm} \\ \equiv 0 \leq i < |eventos| \yLuego (eventos[i] \lor \neg eventos[i]) \wedge 
0 \leq i+1 \leq |eventos| \yLuego \\\textbf{res}= recursos  * 
 \prod_{j=0}^{\textbf{i-1}} \IfThenElse{eventos[j]}{apuesta.c * pago.c}{apuesta.s * pago.s} \vspace{2mm} \\ \equiv  0 \leq i < |eventos| \yLuego
0 \leq i+1 \leq |eventos| \yLuego \\\textbf{res}= recursos  * 
 \prod_{j=0}^{\textbf{i-1}} \IfThenElse{eventos[j]}{apuesta.c * pago.c}{apuesta.s * pago.s}
 \vspace{2mm} \\ \equiv 
0 \leq i+1 \leq |eventos| \yLuego \\\textbf{res}= recursos  * 
 \prod_{j=0}^{\textbf{i-1}} \IfThenElse{eventos[j]}{apuesta.c * pago.c}{apuesta.s * pago.s}$ 
\vspace{5mm}
\\
Asi que, tenemos lo siguiente: 
\vspace{5mm}
\\
$WP(S1,I) \equiv WP(\IfThenElse{B}{S2}{S3};i:=i+1,I) \equiv  \vspace{2mm} \\ 
\{ 0 \leq i+1 \leq |eventos| \yLuego \\\textbf{res}= recursos  * 
 \prod_{j=0}^{\textbf{i-1}} \IfThenElse{eventos[j]}{apuesta.c * pago.c}{apuesta.s * pago.s}\}$
\vspace{5mm}
\\
Debemos ahora, con los calculos hechos, volver al principio, e intentar demostrar la siguiente formula:
 \begin{center}
$\{I \wedge B\} \implies WP(S1,I)$:
 \end{center}

Donde 
\vspace{5mm}
\\
$\{I \wedge B\} \equiv \{0 \leq i < |eventos| \yLuego res = recursos *
 \prod_{j=0}^{i-1} \IfThenElse{eventos[j]}{apuesta.c * pago.c}{apuesta.s * pago.s} \} $
 \vspace{5mm}
\\
Y  WP(S1,I) es el resultado que obtuvimos recien, es decir, 
 \vspace{5mm}
\\
$WP(S1,I) \equiv
\{ 0 \leq i+1 \leq |eventos| \yLuego \textbf{res}= recursos  * 
 \prod_{j=0}^{\textbf{i-1}} \IfThenElse{eventos[j]}{apuesta.c * pago.c}{apuesta.s * pago.s}\}$
\begin{center}
$0 \leq i < |eventos| \implies 0 \leq i+1 \leq |eventos| $ 
\end{center}
Y como lo que esta luego del  es lo mismo en ambos predicados, damos por terminada la demostracion.
 \vspace{15mm}
\\
\underline{$\{I \wedge \neg B\} \implies Q_{c}$:}
 \vspace{5mm}
\\
$\{I \wedge \neg B\} \equiv \{0 \leq i \leq |eventos| \yLuego \vspace{2mm} \\res = recursos *
 \prod_{j=0}^{i-1} \IfThenElse{eventos[j]}{apuesta.c * pago.c}{apuesta.s * pago.s} \wedge i >= |eventos| \}  \vspace{2mm} \\ 
 \equiv \{ i = |eventos| \yLuego res = recursos *
 \prod_{j=0}^{i-1} \IfThenElse{eventos[j]}{apuesta.c * pago.c}{apuesta.s * pago.s}\}  \vspace{2mm} \\ 
 \equiv \{ i = |eventos| \yLuego res = recursos *
 \prod_{j=0}^{|eventos|-1} \IfThenElse{eventos[j]}{apuesta.c * pago.c}{apuesta.s * pago.s}\} \vspace{2mm} \\ 
 \equiv \{ res = recursos * (apuesta.c * pago.c)^{cantApariciones(eventos,T)}*(apuesta.s * pago.s)^{cantApariciones(eventos,F)}\}\vspace{2mm} \\ 
 \equiv Q_{c}. \vspace{5mm}  $\\
\vspace{5mm} 
 Por lo que queda demostrada la tercera implicación. 

Hemos demostrado que el ciclo es parcialmente correcto respecto de su especificación. Para ahora demostrar que el ciclo termina, usaremos una funcion variante $f_{v}$ y probaremos lo siguiente:
\begin{itemize}
	\item $\{ I \wedge B \wedge f_{v} = V_{0} \}\hspace{0.1 cm} S\hspace{0.1 cm} \{ f_{v} < V_{0}\} $
	\item $\{I \wedge f_{v} \leq 0 \} \implies \{\neg B\}$
\end{itemize}

Donde $f_{v} =|eventos| - i$, e $I, B \hspace{0.1 cm} y\hspace{0.1 cm} S$ los mismos que en el apartado anterior.
 \vspace{5mm}
\\
\underline{$\{ I \wedge B \wedge f_{v} = V_{0} \}\hspace{0.1 cm} S\hspace{0.1 cm} \{ f_{v} < V_{0}\} $:}
 \vspace{5mm}
\\
Demostrar este punto, nuevamente es demostrar que
$\{ I \wedge B \wedge f_{v} = V_{0} \} \implies WP(S,f_{v} < V_{0}\}$.
 \vspace{5mm}
\\
\textbf{Calculamos $WP(S,f_{v} < V_{0}\} $}:
 \vspace{5mm}
\\
$WP(\IfThenElse{B}{S2}{S3};i:= i+1,f_{v} < V_{0}) \equiv \vspace{2mm} \\ WP(\IfThenElse{B}{S2}{S3},WP(i:= i+1,f_{v} < V_{0}))  \equiv 
\vspace{2mm} \\ WP(\IfThenElse{B}{S2}{S3},|eventos| -i-1 < V_{0}) \equiv  
\vspace{2mm} \\ (B \wedge WP(S2,|eventos| -i-1 < V_{0})) \lor (\neg B \wedge WP(S3,|eventos| -i-1 < V_{0})) \equiv 
\vspace{2mm} \\ (B \wedge eventos| -i-1 < V_{0}) \lor  (\neg B \wedge eventos| -i-1 < V_{0}) \equiv 
\vspace{2mm} \\  (B \lor \neg B) \wedge |eventos| -i-1 < V_{0} \equiv  |eventos| -i-1 < V_{0} .
$ 
 \vspace{5mm}
\\
Luego, 
 \vspace{5mm}
\\
$
\{ I \wedge B \wedge f_{v} = V_{0} \} \equiv 
\{0 \leq i \leq |eventos| \yLuego res = recursos *
 \prod_{j=0}^{i-1} \IfThenElse{eventos[j]}{apuesta.c * pago.c}{apuesta.s * pago.s} \wedge |eventos| - i = V_{0}  \} \implies \{ |eventos| -i-1 < V_{0} \} 
$
 \vspace{5mm}
\\
Ya que, si
$
 |eventos| - i = V_{0}, \hspace{0.2cm}$ se tiene que $  \hspace{0.2cm}|eventos| -i-1 < V_{0} 
$
 \vspace{5mm}
\\
\underline{$\{I \wedge f_{v} \leq 0 \} \implies \{\neg B\}$:}
 \vspace{5mm}
\\
$
\{I \wedge f_{v} \leq 0 \} \equiv 
\vspace{2mm} \\ \{ 0 \leq i \leq |eventos| \yLuego res = recursos *
 \prod_{j=0}^{i-1} \IfThenElse{eventos[j]}{apuesta.c * pago.c}{apuesta.s * pago.s} \wedge |eventos| - i \leq 0\} \equiv 
 \vspace{2mm} \\ 
 \{ i = |eventos| \yLuego\vspace{2mm}  \\   res = recursos *
 \prod_{j=0}^{|eventos|-1} \IfThenElse{eventos[j]}{apuesta.c * pago.c}{apuesta.s * pago.s} \} \implies \neg \{  i < |eventos| \} \equiv \{ \neg B\}. 
$
 \vspace{5mm}
\\
Finalmente, el ciclo termina y es correcto respecto de su especificación.
 \vspace{5mm}
\\


Ahora bien, ¿podemos decir cual es la precondición mas debil del ciclo? ¿Qué tenemos que hacer para probar que 
\vspace{5mm}
\\ $\{Pre\}res:= recursos; i:=0; \textbf{while}\{Post\}$ es válida?
\begin{enumerate} \setlength\itemsep{0cm}
	\item $ \{Pre\} \implies WP(res:=recursos;i:=0,P_{c}) $
	\item $P_{c} \implies WP(while,Q_{c}) $
	\item $Q_{c} \implies \{Post\} $
\end{enumerate}

El punto 1 se demuestra observando que 
\vspace{2mm} \\
$\{Pre\} \equiv \{apuesta.c + apuesta.s = 1 \wedge pago.c > 0 \wedge pago.s > 0 \wedge apuesta.c > 0 \wedge apuesta.s > 0 \wedge recurso > 0 \}$
\vspace{2mm} \\
implica 
\vspace{2mm} \\
$WP(res:=recursos; i:=0, P_{c}) \equiv \{ i=0 \wedge res = recursos\}$
\vspace{2mm} \\
Como los recursos iniciales son positivos, el punto 1 es evidentemente correcto.
\vspace{5mm} \\


Los puntos 2 y 3 son igualdades de terminos, asi que la implicación es valida en ambos casos.\vspace{5mm} \\ Luego, por Monotonía:
\vspace{5mm} \\
$ \{Pre\} \implies WP(res:=recursos;i:=0;while,Post) $
\vspace{5mm} \\ Que era lo que queríamos demostrar.

\end{align*}











\end{document}
